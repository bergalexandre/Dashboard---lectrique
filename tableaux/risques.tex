\begin{tabularx}{\linewidth}{
    |>{\let\newline\\\hsize=0.40\hsize}X|% 10% of 4\hsize 
    >{\hsize=0.25\hsize}X|% 30% of 4\hsize
    >{\hsize=0.25\hsize}X|% 30% of 4\hsize 
    >{\centering\arraybackslash\hsize=0.1\hsize}X|% 30% of 4\hsize 
       % sum=0.2\hsize for 4 columns
  }
    \hline
    \textbf{Risque} & \textbf{Prévention} & \textbf{Gestion} & \textbf{Priorité}\\\hline
  % Mauvaises connexion des phases du moteur & revues des pairs avant exécution & Rebobinage du moteur ou utilisation du moteur de remplacement & HAUTE\\\hline
  
    Moteur V0 non fonctionnel & Mettre les bouchées double afin d'avoir un prototype & Se limiter à un moteur commercial ou faire usiner à l'extérieur & HAUTE\\\hline
    
    Moteur V0 / commercial non caractérisés pour l'échéancier du 6 octobre & Intégration partielle au véhicule & Communication active avec les équipes interdépendantes & MOYENNE\\\hline
    
    
    
    %Moteur qui ne respecte pas les spécifications acceptables & Faire simulations, calculs, revues de conception avec experts & Achat d'un moteur plan B avec date limite & HAUTE\\\hline
    

    %Mauvaise conception système onduleur & Revue de conception avec experts, validation V1, validation PCB contrôle, attribuer MVP & acquisition onduleur (att. Budget/Temps) & HAUTE\\\hline
    
    MCU pas assez rapide pour controler le moteur Axi & Planifié l'utilisation d'un MCU plus puissant & Être prêt à changer du MCU si le moteur AXI devient la solution & HAUTE\\\hline
    
    Manque de temps en raison des APPs qui recommencent & Prévu dans la planification & Suivre la planification & MOYENNE\\\hline
    
    Commande tardive des connecteurs & Communication directe avec commanditaire & Commande faite par l'équipe au lieu du commanditaire & MOYENNE\\\hline

\end{tabularx}
  
  
% Template pour le tableau des risques de la semaine : 

% Risque : le risque de la semaine
% Mitigation: comment mitiger le risque pour réduire ses conséquences/occurances/etc.
% Conséquence : La ou les conséquences si le risque survient
% Priorité : Priorité du risque sur une échelle de 1 à 5.  5 étant + prioritaire. La priorité est basé sur les conséquences la probablité d'occurence, etc.  
  
%  \textbf{Risque} & \textbf{Mitigation} & \textbf{Conséquence} & \textbf{Priorité}\\\hline
%    Risque & Mitigation & Conséquence & Priorité\\\hline
%    Risque & Mitigation & Conséquence & Priorité\\\hline
%    Risque & Mitigation & Conséquence & Priorité\\\hline
%    Risque & Mitigation & Conséquence & Priorité\\\hline
%    Risque & Mitigation & Conséquence & Priorité\\\hline
%    Risque & Mitigation & Conséquence & Priorité\\\hline
%    Risque & Mitigation & Conséquence & Priorité\\\hline
%    Risque & Mitigation & Conséquence & Priorité\\\hline
%    Risque & Mitigation & Conséquence & Priorité\\\hline
%    Risque & Mitigation & Conséquence & Priorité\\\hline
%    Risque & Mitigation & Conséquence & Priorité\\\hline
%    Risque & Mitigation & Conséquence & Priorité\\\hline