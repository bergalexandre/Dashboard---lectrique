\begin{tabularx}{\linewidth}{
    |>{\let\newline\\\hsize=0.40\hsize}X|% 10% of 4\hsize 
    >{\hsize=0.25\hsize}X|% 30% of 4\hsize
    >{\hsize=0.25\hsize}X|% 30% of 4\hsize 
    >{\centering\arraybackslash\hsize=0.1\hsize}X|% 30% of 4\hsize 
       % sum=0.2\hsize for 4 columns
  }
    \hline
    \textbf{Risque} & \textbf{Prévention} & \textbf{Gestion} & \textbf{Priorité}\\\hline
    Moteur V0 peu fiable & Utiliser V0 comme plateforme de validation seulement & Se limiter à un moteur commercial et faire usiner un moteur V1 à l'extérieur & HAUTE\\\hline
    
    Contrôle moteur non fonctionnel & Réattribution des ressources, rencontre avec un expert, transfert des connaissances & Contrôle open-loop pour les démos & HAUTE\\\hline

    MCU pas assez rapide pour contrôler le moteur commercial & Planifier l'utilisation d'un MCU plus puissant & Être prêt à changer de MCU si le moteur commercial devient la solution & HAUTE\\\hline

    Indisponibilité d'usinage externe du rotor du moteur V1 & Usiner le rotor à l'interne & Être prêt à machiner plusieurs rotors pour rencontrer les tolérances & MOYENNE\\\hline
    
    Manque de temps en raison des APP qui recommencent & Prévu dans la planification & Suivre la planification & MOYENNE\\\hline
    
    BMS IDNEO non fonctionnel / prend beaucoup de temps à tester &  Avoir un autre BMS déjà fonctionnel & Poser beaucoup de questions à IDNEO & BASSE\\\hline

\end{tabularx}

% Mauvaises connexion des phases du moteur & revues des pairs avant exécution & Rebobinage du moteur ou utilisation du moteur de remplacement & HAUTE\\\hline

% Moteur qui ne respecte pas les spécifications acceptables & Faire simulations, calculs, revues de conception avec experts & Achat d'un moteur plan B avec date limite & HAUTE\\\hline

% Mauvaise conception système onduleur & Revue de conception avec experts, validation V1, validation PCB contrôle, attribuer MVP & acquisition onduleur (att. Budget/Temps) & HAUTE\\\hline
  
% Template pour le tableau des risques de la semaine : 

% Risque : le risque de la semaine
% Mitigation: comment mitiger le risque pour réduire ses conséquences/occurances/etc.
% Conséquence : La ou les conséquences si le risque survient
% Priorité : Priorité du risque sur une échelle de 1 à 5.  5 étant + prioritaire. La priorité est basé sur les conséquences la probablité d'occurence, etc.  
  
%  \textbf{Risque} & \textbf{Mitigation} & \textbf{Conséquence} & \textbf{Priorité}\\\hline
%    Risque & Mitigation & Conséquence & Priorité\\\hline
%    Risque & Mitigation & Conséquence & Priorité\\\hline
%    Risque & Mitigation & Conséquence & Priorité\\\hline
%    Risque & Mitigation & Conséquence & Priorité\\\hline
%    Risque & Mitigation & Conséquence & Priorité\\\hline
%    Risque & Mitigation & Conséquence & Priorité\\\hline
%    Risque & Mitigation & Conséquence & Priorité\\\hline
%    Risque & Mitigation & Conséquence & Priorité\\\hline
%    Risque & Mitigation & Conséquence & Priorité\\\hline
%    Risque & Mitigation & Conséquence & Priorité\\\hline
%    Risque & Mitigation & Conséquence & Priorité\\\hline
%    Risque & Mitigation & Conséquence & Priorité\\\hline