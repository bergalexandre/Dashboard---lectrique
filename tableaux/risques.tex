\begin{tabularx}{\linewidth}{
    |>{\let\newline\\\hsize=0.40\hsize}X|% 10% of 4\hsize 
    >{\hsize=0.25\hsize}X|% 30% of 4\hsize
    >{\hsize=0.25\hsize}X|% 30% of 4\hsize 
    >{\centering\arraybackslash\hsize=0.1\hsize}X|% 30% of 4\hsize 
       % sum=0.2\hsize for 4 columns
  }
    \hline
    \textbf{Risque} & \textbf{Prévention} & \textbf{Gestion} & \textbf{Priorité}\\\hline
   Mauvaises connexion des phases du moteur & revues avec es pairs avant exécution & Rebobinage du moteur ou utilisation du moteur de remplacement & HAUTE\\\hline
    
    Moteur qui ne respecte pas les spécifications acceptables & Faire simulations, calculs, revues de conception avec experts & Achat d'un moteur plan B avec date limite & HAUTE\\\hline
    
    Transmission qui ne fonctionne pas & Réception hative des pièces pour validation & Acquisition réducteur vitesse (att. Budget/Temps) & MOYENNE\\\hline
    
    Mauvaise conception système onduleur & Revue de conception avec experts, validation V1, validation PCB contrôle, attribuer MVP & acquisition onduleur (att. Budget/Temps) & HAUTE\\\hline
    
    Bris système pendant phases de test et validation & Faire simulations, calculs, revue de conception avec experts et planification des tests et analyse des risques & Acquisition composants supplémentaires & MOYENNE\\\hline

    Gestion de l'équipe communication & Réunion bihebdomadaire technique et réunion hebdomadaire SteerCo. & Réattribution des ressources, modification des priorités, réévaluation des risques et du temps & MOYENNE\\\hline
    
    Gestion des tâches de l'équipe & Réunion bihebdomadaire technique et calendrier détaillée des tâches & Acquisition de composants (att. Budget/Temps) & MOYENNE\\\hline

    Manque budget / temps / ressources & Communication interdépendante des équipes & Tirer l'échéancier vers le début de la session pour identifier les problèmes & HAUTE\\\hline
    
  \end{tabularx}
  
  
% Template pour le tableau des risques de la semaine : 

% Risque : le risque de la semaine
% Mitigation: comment mitiger le risque pour réduire ses conséquences/occurances/etc.
% Conséquence : La ou les conséquences si le risque survient
% Priorité : Priorité du risque sur une échelle de 1 à 5.  5 étant + prioritaire. La priorité est basé sur les conséquences la probablité d'occurence, etc.  
  
%  \textbf{Risque} & \textbf{Mitigation} & \textbf{Conséquence} & \textbf{Priorité}\\\hline
%    Risque & Mitigation & Conséquence & Priorité\\\hline
%    Risque & Mitigation & Conséquence & Priorité\\\hline
%    Risque & Mitigation & Conséquence & Priorité\\\hline
%    Risque & Mitigation & Conséquence & Priorité\\\hline
%    Risque & Mitigation & Conséquence & Priorité\\\hline
%    Risque & Mitigation & Conséquence & Priorité\\\hline
%    Risque & Mitigation & Conséquence & Priorité\\\hline
%    Risque & Mitigation & Conséquence & Priorité\\\hline
%    Risque & Mitigation & Conséquence & Priorité\\\hline
%    Risque & Mitigation & Conséquence & Priorité\\\hline
%    Risque & Mitigation & Conséquence & Priorité\\\hline
%    Risque & Mitigation & Conséquence & Priorité\\\hline