{\large \textbf{Moteur / Transmission}}
\smallskip

\begin{tabularx}{\linewidth}{
    |>{\centering\hsize=0.25\hsize}X|%
    >{\centering\hsize=0.25\hsize}X|%
    >{\hsize=2.75\hsize}X|%
    >{\hsize=0.75\hsize}X|%
  }
    \hline
    \textbf{Planifié}
        &\textbf{Progrès}
        &\multicolumn{1}{>{\centering\hsize=2.5\hsize}X|}{\textbf{Objectif}}
        &\multicolumn{1}{>{\centering\hsize=0.75\hsize}X|}{\textbf{Responsable}}
    \\\hline
    100\% & 100\% & Fabriquer le gabarit de montage des laminations stator (Eau) & Francois Paquette\\\hline
    0\% & 0\% & Fabriquer le gabarit de montage des laminations stator (EDM) & Francois Paquette\\\hline
    100\% & 50\% & Découpe lamination & Francois Paquette \\\hline
    50\% & 20\% & Machinage rotor & Francois Paquette \\\hline
    100\% & 85\% & Batterie V0 caractérisée et assemblée & Francois Paquette\\\hline
    66\% & 66\% & Assemblage numérique de la transmission (SolidWorks) & Thomas Chagnon\\\hline
    85\% & 85\% & Choix des roulements à billes & Xavier Morin\\\hline
    25\% & 0\% & Calcul numérique: Validations FEA & Xavier Morin\\\hline
\end{tabularx}
\medskip

{\large \textbf{Bloc batterie}}
\smallskip

\begin{tabularx}{\linewidth}{
    |>{\centering\hsize=0.25\hsize}X|%
    >{\centering\hsize=0.25\hsize}X|%
    >{\hsize=2.75\hsize}X|%
    >{\hsize=0.75\hsize}X|%
  }
    \hline
    \textbf{Planifié}
        &\textbf{Progrès}
        &\multicolumn{1}{>{\centering\hsize=2.5\hsize}X|}{\textbf{Objectif}}
        &\multicolumn{1}{>{\centering\hsize=0.75\hsize}X|}{\textbf{Responsable}}
    \\\hline
    60\% & 50\% & Assemblage et test du BMS V1 & Loïc Poirier, Jérôme Gelé
    \\\hline
    100\% & 80\% & Approbation du protocole d'assemblage et assemblage bloc batterie V0 & François Paquette
    \\\hline
\end{tabularx}
\medskip

{\large \textbf{Onduleur}}
\smallskip

\begin{tabularx}{\linewidth}{
    |>{\centering\hsize=0.25\hsize}X|%
    >{\centering\hsize=0.25\hsize}X|%
    >{\hsize=2.75\hsize}X|% 
    >{\hsize=0.75\hsize}X|%
  }
    \hline
    \textbf{Planifié}
        &\textbf{Progrès}
        &\multicolumn{1}{>{\centering\hsize=2.5\hsize}X|}{\textbf{Objectif}}
        &\multicolumn{1}{>{\centering\hsize=0.75\hsize}X|}{\textbf{Responsable}}
    \\\hline
    100\% & 100\% & PCB V1 DC/DC & Vincent Bonneau
    \\\hline
    85\% & 85\% & Assemblage et tests PCB onduleur V1 & Marc-Antoine Dubreuil
    \\\hline
    25\% & 10\% & Conception PCB onduleur V2 & Marc-Antoine Dubreuil
    \\\hline
\end{tabularx}
\medskip

{\large \textbf{Instrumentation}}
\smallskip

\begin{tabularx}{\linewidth}{
    |>{\centering\hsize=0.25\hsize}X|%
    >{\centering\hsize=0.25\hsize}X|%
    >{\hsize=2.75\hsize}X|%
    >{\hsize=0.75\hsize}X|%
  }
    \hline
    \textbf{Planifié}
        &\textbf{Progrès}
        &\multicolumn{1}{>{\centering\hsize=2.5\hsize}X|}{\textbf{Objectif}}
        &\multicolumn{1}{>{\centering\hsize=0.75\hsize}X|}{\textbf{Responsable}}
    \\\hline
    5\% & 5\% & Design du PCB de DEL & Joël Grégoire-Lagueux \\\hline
    5\% & 5\% & Design du PCB de contrôle & Vincent Bonneau \\\hline
    100\% & 5\% & Commande des pièces & Joël Grégoire-Lagueux \\\hline
    100\% & 90\% & Réalisation d'un plan de tests & Joël Grégoire-Lagueux \\\hline
    % 0\% & 0\% & Assemblage,contrôle et tests & Joël Grégoire-Lagueux \\\hline
\end{tabularx}