\Large\begin{tabularx}{\linewidth}{
    |>{\hsize=3.0\hsize}X|% 10% of 4\hsize 
    >{\hsize=0.75\hsize}X|% 30% of 4\hsize
    >{\hsize=0.5\hsize}X|% 30% of 4\hsize 
    >{\hsize=0.5\hsize}X|% 30% of 4\hsize 
    >{\hsize=0.25\hsize}X|% 30% of 4\hsize 
       % sum=0.2\hsize for 4 columns
  }
    \hline
    tâches & Système & Responsable & État & Heure\\\hline
    Modélisation préliminaire (courbes préliminaires) & Moteur et Transmission & François Paquette & En Cours & 3.0\\\hline
    Choix topologie  S P offrant le plus de flexibilité  & Batterie BMS & François Paquette & En Cours & 3.0\\\hline
    Créer fichier comparaison des différentes cellules en fonction de la quantité d'énergie à mettre à bord & Batterie BMS & François Paquette & En Cours & 15.0\\\hline
    Vérifier réglementation sur le design batterie & Batterie BMS & François Paquette & Terminé & 2.0\\\hline
    Étude élémentaire sur refroidissement light weight & Batterie BMS & François Paquette & En Cours & 2.0\\\hline
    Étude préliminaire des technologies de BMS & Batterie BMS & Loïc Poirier & Terminé & 6.0\\\hline
    Modélisation préliminaire (courbes préliminaires) & Moteur et Transmission & Gabriel Cabana & En Cours & 7.0\\\hline
    Sélection du type de moteur selon l'efficacité relative & Moteur et Transmission & Gabriel Cabana & En Cours & 9.0\\\hline
    Créer modèle Mtalab et support modélisation D.T. transmission avec 3 scénarios, Direct Drive, 1 ratio réduction et 2 ratios de réduction & Moteur et Transmission & Thomas Chagnon & En Cours & 3.0\\\hline
    Sortir et identifier les équations d'inertie, les relations de masse équivalente et les paramètres mécaniques roue \& Transmission & Moteur et Transmission & Thomas Chagnon & En Cours & 3.0\\\hline
  \end{tabularx}
     