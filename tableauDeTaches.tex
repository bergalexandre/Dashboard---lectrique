\Large\begin{tabularx}{\linewidth}{
    |>{\hsize=2.0\hsize}X|% 10% of 4\hsize 
    >{\hsize=1.2\hsize}X|% 30% of 4\hsize
    >{\hsize=0.6\hsize}X|% 30% of 4\hsize 
       % sum=0.2\hsize for 4 columns
  }
    \hline
    tâches & Système & Responsable & État & Heure\\\hline
    Créer modèle Mtalab et support modélisation D.T. transmission avec 3 scénarios, Direct Drive, 1 ratio réduction et 2 ratios de réduction & Moteur_et_Transmission & Thomas Chagnon & En Cours & 3.0\\\hline
    Sortir et identifier les équations d'inertie, les relations de masse équivalente et les paramètres mécaniques roue \& Transmission & Moteur_et_Transmission & Thomas Chagnon & En Cours & 3.0\\\hline
    Identification des paramètres pour la modélisation (Résistances, Constantes, Surfaces, Convection,Conduction et Rayonnement & Transferts_Thermiques & Xavier Morin & En Cours & 4.0\\\hline
    Bloc Batterie Modélisation & Simulateur & Xavier Morin & En Cours & 2.0\\\hline
    Bloc Transmission Modélisation & Simulateur & Xavier Morin & En Cours & 1.0\\\hline
    Génération courbes Torque mot et RPM mot en fonction de la vitesse véhicule et accélération véhicule & Moteur_et_Transmission & Xavier Morin & Terminé & 4.0\\\hline
    Sortir et identifier les équations d'inertie, les relations de masse équivalente et les paramètres mécaniques roue \& Transmission & Moteur_et_Transmission & Xavier Morin & En Cours & 3.0\\\hline
      &   &   &   &  \\\hline
      &   &   &   &  \\\hline
      &   &   &   &  \\\hline
  \end{tabularx}
     